% Load your needed packages and other commands of yours here:
%\usepackage{} % ... note that old .sty files can be included here

















%--------------------------------------------------------------------------
% Do NOT edit the rest of the preample UNLESS YOU KNOW WHAT YOU'RE DOING!
%--------------------------------------------------------------------------


\usepackage{amsmath,amssymb,amstext} % Lots of math symbols and environments
\usepackage{graphicx} % For including graphics 

\usepackage{nomentbl} 
\makenomenclature 

\usepackage{ifpdf}




\let\paragraph\oldparagraph
\let\subparagraph\oldsubparagraph

\usepackage{titlesec}
% Remake title styles
\titleclass{\chapter}{straight}
\titleformat{\chapter}[display]
{\normalfont\huge\bfseries}{}{0pt}{\Huge}

\titleformat{\chapter}{\bfseries\LARGE}{\thechapter.~ }{0pt}{}
\titlespacing*{\chapter}{0pt}{0pt}{0pt}

\titleformat{\section}{\bfseries\Large}{\thesection.~}{0pt}{}
\titlespacing*{\section}{0pt}{0pt}{0pt}


\titleformat{\subsection}{\bfseries\large}{\thesubsection.~}{0pt}{}

\titleformat{\subsubsection}{\bfseries\normalsize}{\thesubsubsection.~}{0pt}{}

% Adds space between paragraphs
\setlength{\parindent}{2em}

% Removes breaks from include. To use that functionality use include*{chapter}
\usepackage{newclude}

%%% redo margins

% Main text box is the width/height of the paper minus 2in
\setlength{\textwidth}{6.5in}
\setlength{\textheight}{8.75in} % + footskip (0.25) = 9.00in

% Only the left and top margins can be defined
% 1in is added to given values
\setlength{\topmargin}{0.0in}
\setlength{\evensidemargin}{0.0in}
\setlength{\oddsidemargin}{0.0in}

% The top margin is also affected by the headhight and headsep
\setlength{\headheight}{0.0in}
\setlength{\headsep}{0.0in}

% The bottom margin is also affected by the footskip
% i.e. the length of the page has the following markers:
% top of text at 1.00in, bottom of text at 9.75in,
% page number at 10.00in, bottom of page at 11.00in
\setlength{\footskip}{0.25in}

% Making sure absolutely nothing passes the right margin
\setlength{\hfuzz}{0.0pt}

% Changing where the page numbers go on pages that aren't the first
% page of a chapter.

%\pagestyle{myheadings}
%\pagestyle{plain}
%\markright{}





%%%%
% Line spacing
\renewcommand{\baselinestretch}{1.9} % this is the default line space setting

% By default, each chapter will start on a recto (right-hand side)
% page.  We also force each section of the front pages to start on 
% a recto page by inserting \cleardoublepage commands.
% In many cases, this will require that the verso page be
% blank and, while it should be counted, a page number should not be
% printed.  The following statements ensure a page number is not
% printed on an otherwise blank verso page.
\let\origdoublepage\cleardoublepage
\newcommand{\clearemptydoublepage}{%
  \clearpage{\pagestyle{empty}\origdoublepage}}
\let\cleardoublepage\clearemptydoublepage

% To make space between lines
\usepackage{setspace}

%
\usepackage{tocloft}
\renewcommand{\cftsecfont}{\normalfont}

% Align all entries in TOC
\renewcommand{\cftsecindent}{0pt}
\renewcommand{\cftsubsecindent}{0pt}
\renewcommand{\cftsubsubsecindent}{0pt}


\renewcommand{\cfttoctitlefont}{\Huge\bfseries}

\renewcommand{\cftchapfont}{\normalfont}
\renewcommand{\cftsecfont}{\normalfont}
\renewcommand{\cftsubsecfont}{\normalfont}
\setlength{\cftbeforetoctitleskip}{0pt}

%\renewcommand{\cftnotnumb}[1]{\bfseries #1}
\renewcommand{\cftchapfont}{\normalfont}
\renewcommand{\cftchappagefont}{\normalfont}

\renewcommand{\cftsecleader}{\cftdotfill{\cftdotsep}}

\renewcommand{\cftpartfont}{\normalfont}
\renewcommand{\cftpartpagefont}{\normalfont}

%\usepackage{showframe}

\usepackage{datetime,scrdate}
\newdateformat{nwfmt}{\THEDAY\ \monthname,\ \THEYEAR}

\newcommand{\datepublish}{\newdate{mydate}{$book.date.day$}{$book.date.month$}{$book.date.year$}% day/month/year
                          \nwfmt\displaydate{mydate}}
                          
% APA citation
\usepackage{apacite}
\bibliographystyle{apacite}







%
\AtBeginDocument{
\hypersetup{pdftitle=$book.title$} % Set the PDF's title to your title
\hypersetup{pdfauthor=$author$} % Set the PDF's author to your name
}
